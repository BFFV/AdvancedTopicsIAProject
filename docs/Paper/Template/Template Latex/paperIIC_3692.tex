\documentstyle[nips01e,graphicx,color,subfigure]{article}

\title{Titulo del Documento}

\author{Nombre Apellido\\
Pontificia Universidad Cat\'olica de Chile\\
{\it login@ing.puc.cl} \\
}

\begin{document}

\graphicspath{{figs/}{eps/}}

\maketitle

\begin{abstract}
Resumen del documento incluyendo los aspectos m\'as relevantes y
principales resultados. No m\'as de 300 palabras.
\end{abstract}

\section{Introducci�n}
En esta secci\'on presentar antecedentes relevantes para el
trabajo, adem\'as de las motivaciones principales.

En el p\'arrafo final de esta secci\'on dar una breve
descripci\'on de las principales partes del documento. Por
ejemplo, un texto en ingl\'es describiendo este p\'arrafo final
podr\'ia ser:

This paper is organized as follows. Section 2 describes previous
work and gives background information about the operation of yara
yara.... Section 3 presents the methods proposed in this paper and
discusses the results of the experiments. Finally, Section 4
presents the main conclusions of this work.

\section{Antecedentes} \label{background}
\subsection{Aspectos Te\'oricos}
Esta secci\'on se puede usar para mostrar material te\'orico
relevante para enteder el trabajo.

\subsection{Trabajos Relacionados} \label{previous_work}
Descripci\'on de trabajos previos.

\section{M\'etodo Propuesto}
Describir los aspectos m\'as relevantes de este trabajo.
Importante incluir referencias
\cite{Ejemplo1:1989}\cite{Ejemplo2:1993}.

\section{Resultados}
Describir los resultados alcanzados comparando con soluciones
alternativas.

\section{Conclusions} \label{conclusions}
Principales conclusiones de este trabajo y comentarios sobre
posibles mejoramientos y trabajos futuros.

\small{
\bibliographystyle{plain}
\bibliography{references}
}
\end{document}
